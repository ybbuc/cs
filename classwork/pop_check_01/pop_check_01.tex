\documentclass[a4paper,answers,addpoints]{exam}
\usepackage{color}
\usepackage{graphicx}
\colorfillwithlines{}
\pagestyle{headandfoot}
\firstpageheader{CS + Robotics}{\textbf{Pop Check 01}}{October 8, 2022}
\boxedpoints
\pointsinmargin
\begin{document}
\makebox[0.5\textwidth]{Name:\enspace\hrulefill}
\vspace{20pt}
\begin{questions}
    \question[2] This is a \fillin[file].
    \includegraphics[width=24px]{folder.png}
    \question[2] This is a \fillin[folder].
    \includegraphics[width=24px]{file.png}
    \question[4] ``\texttt{cute\_cat.png}'' is the \fillin[filename] of an image. ``\texttt{.png}'' is the \fillin[file extension].
    \question[2] The computer understands that files ending with \fillin[\texttt{.py}] are Python files.
    \question[2] The \fillin[command line] is a user interface that is navigated by typing commands at prompts, instead of using a mouse.
    \question[5] What command should you use to run a Python file? (Choose a name for the file.)
    \begin{solutionorlines}[50pt]
        \texttt{python hello\_world.py}
    \end{solutionorlines}
    \question[5] Write the code necessary to display ``\texttt{Hello world!}'' using Python.
    \begin{solutionorlines}[50pt]
    \texttt{print("Hello world!")}
    \end{solutionorlines}
\end{questions}
\vspace{35pt}
\begin{center}
    \gradetable[h][questions]
\end{center}
\end{document}
