\documentclass[a4paper,addpoints]{exam}
\usepackage{color}
\usepackage{graphicx}
\colorfillwithlines{}
\pagestyle{headandfoot}
\firstpageheader{CS + Robotics}{\textbf{Classwork 05}}{Introducing Lists}
\boxedpoints
\pointsinmargin
\begin{document}
\makebox[0.5\textwidth]{Name:\enspace\hrulefill}
\vspace{20pt}
\begin{questions}
    \question[10] Read the introduction to chapter 4 (pg. 49) of \textit{Python Crash Course}.
        What does looping a list in Python allow us to do?
        \begin{solutionorlines}[\stretch{1}]
        \end{solutionorlines}
    \question[20] Read the section ``What is a list?'' (pg. 33-36), then answer the following questions and do the following tasks.
        \begin{parts}
            \part What is a list?
                \begin{solutionorlines}[\stretch{1}]
                \end{solutionorlines}
            \part What is the start of the index position?
                \begin{solutionorlines}[\stretch{1}]
                \end{solutionorlines}
            \part Open the Python interpreter in the command line and create a list of strings as the example in the book with at least four elements. 
            \part Print the first and the last item of the list and use the method \texttt{title()} on them.
            \part Use an f-string to create a message with two of the elements in your list, then print this message.  
            \part Show all these tasks the teacher. (Otherwise you will lose points.)
        \end{parts}
        \question[10] Create a Python file named \texttt{lists.py} and do all the tasks in the section ``TRY IT YOURSELF'' on pg. 36.
            Then add this file to your GitLab repository.
        \textbf{Note: Add comments to your Python file to indicate which task is which.}
        \question[20] Read the section ``Changing, Adding, and Removing Elements'' (pg. 36-42), then answer the following question and do the following tasks. 
        \begin{parts}
            \part What list methods did you learn in this section?
                \begin{solutionorlines}[\stretch{1}]
                \end{solutionorlines}
            \part Open the Python interpreter in the command line and create a list of strings with at least three elements.
            \part Modify the second element of your list.
            \part Add a new element at the end of your list using the correct method.
            \part Create a new empty list and then append 3 elements using the correct method and then print that list.
        \clearpage
            \part Add a new element at the beginning of your list using the correct method.
            \part Remove the last item of your list using the correct statement.
            \part Use the correct method to remove the last item of your list and save the value into a new variable called \texttt{popped\_item}.
            \part Use the \texttt{remove()} method to remove one element of your list by value.
            \part Show all these tasks the teacher. (Otherwise you will lose points.)
        \end{parts}
        \question[10] In your Python file named \texttt{lists.py} do the all the tasks in the section ``TRY IT YOURSELF'' on pg. 42.
            Then add this file to your GitLab repository.
            \textbf{Note: Add comments to your Python file to indicate which task is which.}
            \question[10] Read the section ``Organizing a List'' (pg. 43-46), then answer the questions.
            \begin{parts}
                \part What method do we use to order a list?
                    \begin{solutionorlines}[\stretch{1}]
                    \end{solutionorlines}
                \part What method do we use to find the length of a list?
                    \begin{solutionorlines}[\stretch{1}]
                    \end{solutionorlines}
            \end{parts}
        \question[10] In your Python file named \texttt{lists.py} do the all the tasks in the section ``TRY IT YOURSELF'' on pg. 46.
            Then add this file to your GitLab repository.
            \textbf{Note: Add comments to your Python file to indicate which task is which.}
        \question[10] Read the section ``Avoiding Index Errors When Working with Lists'' (pg. 46-47), then answer the following questions.
        \begin{parts}
            \part What is one of the common errors developers find when working with lists in Python?
                \begin{solutionorlines}[\stretch{1}]
                \end{solutionorlines}
            \part What does this error mean?
                \begin{solutionorlines}[\stretch{1}]
                \end{solutionorlines}
        \end{parts}
\end{questions}
\begin{center}
    \gradetable[h][questions]
\end{center}
\end{document}
