\documentclass[a4paper,addpoints]{exam}
\usepackage{color}
\usepackage{graphicx}
\usepackage{setspace}
\colorfillwithlines{}
\pagestyle{headandfoot}
\firstpageheader{CS + Robotics}{\textbf{Classwork 08}}{Working With Lists}
\boxedpoints
\pointsinmargin
\onehalfspacing
\begin{document}
\makebox[0.5\textwidth]{Name:\enspace\hrulefill}
\vspace{20pt}
\begin{center}
    You will find the answers to these questions in Chapter 4 of \textit{Python Crash Course}, pages 57--60.
\end{center}
\vspace{20pt}
\textbf{Making numerical lists}
\begin{questions}
    \question What is ideal for storing sets of numbers in Python?
    \begin{solutionorlines}[\stretch{1}]
    \end{solutionorlines}

    \vspace{15pt}
    \textbf{Using the \texttt{range()} function}

    \question What does the \texttt{range()} function allow us to do?
    \begin{solutionorlines}[\stretch{1}]
    \end{solutionorlines}

    \question \textit{4-3 Counting to Twenty:} Use a \texttt{for} loop to print the numbers from \texttt{1} to \texttt{20}, inclusive.
    \begin{solutionorlines}[\stretch{1}]
    \end{solutionorlines}

    \vspace{15pt}
    \textbf{Using \texttt{range()} to make a list of numbers}

    \question Which function can we use to convert the results of \texttt{range()} directly into a list?
    \begin{solutionorlines}[1cm]
        \texttt{list()}
    \end{solutionorlines}

    \question If we use the following range to create a list what would be the numbers in the list? \\
    \texttt{range(2, 11, 3)}
    \begin{solutionorlines}[1cm]
        \texttt{2, 5, 8}
    \end{solutionorlines}
    
    \question What range will create a list of odd numbers from 3 to 17?
    \begin{solutionorlines}[1cm]
        \texttt{range(3, 17, 2)}
    \end{solutionorlines}
    \clearpage

    \question A number raised to the third power is called a cube. For example, the cube of 2 is written as \texttt{2**3} in Python.
    Make a list of the first 10 cubes (that is, the cube of each integer from 1 through 10), and use a \texttt{for} loop to print out the value of each cube.
    \begin{solutionorlines}[7cm]
    \end{solutionorlines}
    

    \textbf{Simple statistics with a list of numbers}

    \question Consider the following list of numbers. Fill in the blanks below.
    \begin{verbatim}
numbers = [1, 4, 9, 16, 25]
    \end{verbatim}
    \begin{spacing}{2}
        \texttt{min(numbers) = } \fillin[1][1cm] \\
        \texttt{max(numbers) = } \fillin[25][1cm] \\
        \texttt{sum(numbers) = } \fillin[55][1cm]
    \end{spacing}
    \vspace{15pt}


    \textbf{List comprehensions}

    \question \textit{4-9. Cube Comprehension:} Use a list comprehension to generate a list of the first 10 cubes.
    \begin{solutionorlines}[2cm]
    \end{solutionorlines}


    \textbf{Slicing a List}
    \question Consider the following list of students. Fill in the blanks below. \\
    \texttt{students = ["mary", "jane", "haikun", "sandy", "james", "jeremy"]}
    \begin{spacing}{2}
        \texttt{students[0:2] = } \fillin[1][10cm] \\
        \texttt{students[1:4] = } \fillin[25][10cm] \\
        \texttt{students[:3] = } \fillin[55][10cm] \\
        \texttt{students[3:] = } \fillin[55][10cm]
    \end{spacing}
    \vspace{15pt}

\end{questions}
\end{document}
