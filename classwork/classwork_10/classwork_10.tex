\documentclass[a4paper]{article}
\usepackage{color}
\usepackage{graphicx}
\usepackage{setspace}
\usepackage{menukeys}
\onehalfspacing
\begin{document}
\textbf{\Huge Classwork 10} 
\section{User input in Python}
To get user input in Python, the command we use is \texttt{input()}.
Store the result in a variable, and use it to your heart's content.
Remember that the result you get from the user will be a string, even if they enter a number.
For example:
\begin{verbatim}
name = input("Give me your name: ") 
print("Your name is " + name)
\end{verbatim}
What this will print in the terminal (or the shell, whatever you are running Python in) will be:
\begin{verbatim}
>>> Give me your name: Michele
Your name is Michele
\end{verbatim}
Only after the user presses the \keys{\return{}} key, does the program continue.
\section{Manipulating strings (a few ways)}
What we get from the \texttt{input()} function is a string.
What can you do with it?
\subsection{Make the string into a number}
Let's say you are 100\% positive that the user entered a number.
You can turn the string into an integer with the function \texttt{int()}.
(In a later exercise or two or three there will be questions about what to do when the user does NOT enter a number and you try to do this; for now don't worry about that problem). \\
Here is what this looks like:
\begin{verbatim}
age = input("Enter your age: ") 
age = int(age)
\end{verbatim}
(or, if you want to be more compact with your code)
\begin{verbatim}
age = int(input("Enter your age: "))
\end{verbatim}
In both cases, \texttt{age} will hold a variable that is an integer, and now you can do math with it.
(Note, you can also turn integers into strings exactly in the opposite way, using the \texttt{str()} function.)
\clearpage
\subsection{Do math with strings}
What do I mean by that? I mean, if I want to combine (\textbf{concatenate} is the computer science word for this) strings, all I need to do is add them:
\begin{verbatim}
print("Were" + "wolf") 
print("Door" + "man") 
print("4" + "chan") 
print(str(4) + "chan")
\end{verbatim}
The same works for multiplication:
\begin{verbatim}
print(4 * "test")
\end{verbatim}
but division and subtraction do not work like this.
In terms of multiplication, the idea of multiplying two strings together is not well-defined.
What does it mean to multiply two strings in the first place?
However, it makes sense in a way to specify multiplying a string by a number---just repeat that string that number of times.
Try this in your own program with all the arithmetic operations.
\section{Exercise}
Create a program that asks the user to enter their name and their age.
Print out a message addressed to them that tells them the year that they will turn 100 years old.
\textit{Note: for this exercise, the expectation is that you explicitly write out the year (and therefore be out of date the next year).}
\subsection{Extras}
These are not required.
You'll do these if you're awesome.
\begin{enumerate}
    \item Ask for the current year and use that year in your calculation.
    \item Add on to the previous program by asking the user for another number and printing out that many copies of the previous message. (Hint: order of operations exists in Python.)
    \item Print out that many copies of the previous message on separate lines. (Hint: the string \verb|"\n"| is the same as pressing the \keys{\return{}} key.)
\end{enumerate}
\vspace{\stretch{1}}
\fbox{\fbox{\parbox{0.9\textwidth}{\centering
Note: do this in a file named \texttt{hundred\_years.py}}}}
\end{document}
