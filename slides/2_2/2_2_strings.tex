\documentclass[aspectratio=169]{beamer}
% Themes ---------------------------------------------------------------------
\usetheme{default}
\usefonttheme{professionalfonts}
% Packages -------------------------------------------------------------------
\usepackage{graphicx}
\usepackage{enumitem}
\usepackage{stmaryrd}
\usepackage[most]{tcolorbox}
\usepackage{booktabs}
\usepackage{import}
\usepackage{adjustbox}
\usepackage{MnSymbol}
\usepackage{fancyvrb}
% Tables ---------------------------------------------------------------------
\renewcommand{\arraystretch}{1.3}
% enumitem -------------------------------------------------------------------
\setlist{itemsep=10pt}
% beamer ---------------------------------------------------------------------
\setbeamerfont{caption}{size=\small}
% Title & author -------------------------------------------------------------
\title{2.2 Strings}
\subtitle{Variables and Simple Data Types}
\author{Jakob Wells}

\begin{document}
\date{\today}


\begin{frame}
    \titlepage{}
\end{frame}


\begin{frame}[fragile]{Strings}
    A \textbf{string} is a series of characters.
    Anything inside quotes is considered a string in Python, and we can use single or double quotes around our strings like this:
    \vspace{10pt}
    \begin{Verbatim}
"This is a string."
'This is also a string.'
    \end{Verbatim}
    \vspace{25pt}
    \uncover<2->{This flexibility allows us to use quotes and apostrophes within our strings: \\
    \vspace{10pt}
    \texttt{
'I told my friend, "Python is my favorite language!"' \\
"The language 'Python' is named after Monty Python, not the snake." \\
"One of Python's strengths is its diverse and supportive community."
}}
\end{frame}


\begin{frame}[fragile]{Changing case in a string with methods (I)}
    \begin{Verbatim}
name = "ada lovelace"
print(name.title())
    \end{Verbatim}
    \vspace{15pt}
    Save this file as \textit{name.py}, and then run it. \\
    \vspace{25pt}
    \begin{itemize}[label=--]
        \uncover<2->{\item In this example, the variable \texttt{name} refers to the lowercase string \texttt{"ada lovelace"}.}
        \uncover<3->{\item The method \texttt{title()} appears after the variable in the \texttt{print()} call.}
        \uncover<4->{\item A \textbf{method} is an action that Python can perform on a piece of data.}
        \uncover<5->{\item The dot (\texttt{.}) after \texttt{name} in \texttt{name.title()} tells Python to make the \texttt{title()} method act on the variable \texttt{name}.}
    \end{itemize}
\end{frame}


\begin{frame}{Changing case in a string with methods (II)}
    \begin{itemize}[label=--]
        \item Every method is followed by a set of parentheses, because methods often need additional information to do their work.
        \item The \texttt{title()} function doesn't need any additional information, so its parentheses are empty.
        \item The \texttt{title()} method changes each work to title case, where each word begins with a capital letter.
        \item This is useful because you'll often want to think of a name as a piece of information.
        \item For example, you might want your program to recognize the input values \texttt{Ada}, \texttt{ADA}, and \texttt{ada} as the same name, and display all of them as \texttt{Ada}.
    \end{itemize}
\end{frame}


\begin{frame}[fragile]{Changing case in a string with methods (III)}
    Several other methods are available for dealing with case.
    For example, you can change a string to all uppercase or all lowercase letters like this:
    \vspace{10pt}
    \begin{Verbatim}
name = "Ada Lovelace"
print(name.upper())
print(name.lower())
    \end{Verbatim}
    \vspace{15pt}
    This will display the following:
    \vspace{10pt}
    \begin{Verbatim}
ADA LOVELACE
ada lovelace
    \end{Verbatim}
\end{frame}


\begin{frame}{\texttt{lower()}}
    \begin{itemize}[label=--]
        \item The \texttt{lower()} method is particularly useful for storing data.
        \item Many times you won't want to trust the capitalization that your users provide, so you'll convert strings to lowercase before storing them.
        \item When you want to display the information, just use the case you want.
    \end{itemize}
\end{frame}


\begin{frame}[fragile]{Using variables in strings (I)}
    You might want two variables to represent a first name and last name, and then want to combine those values to display someone's full name:
    \vspace{10pt}
    \begin{Verbatim}
first_name = "ada"
last_name = "lovelace"
full_name = f"{first_name} {last_name}"
print(full_name)
    \end{Verbatim}
    \vspace{25pt}
    To insert a variable's value into a string, place the letter \texttt{f} immediately before the opening quotation mark.
    These strings are called \textbf{f-strings}.
\end{frame}


\begin{frame}[fragile]{Using variables in strings (II)}
    You can use f-strings to create complete messages using the information associated with a variable, as shown here:
    \vspace{10pt}
    \begin{Verbatim}
first_name = "ada"
last_name = "lovelace"
full_name = f"{first_name} {last_name}"
print(f"Hello, {full_name.title()}!")
    \end{Verbatim}
\end{frame}


\begin{frame}[fragile]{Using variables in strings (III)}
    You can also use f-strings to compose a message, and then assign the entire message to a variable:
    \vspace{10pt}
    \begin{Verbatim}
first_name = "ada"
last_name = "lovelace"
full_name = f"{first_name} {last_name}"
message = f"Hello, {full_name.title()}!"
print(message)
    \end{Verbatim}
    \vspace{25pt}
    This makes the final \texttt{print()} call much simpler.
\end{frame}


\begin{frame}[fragile]{Adding whitespace to strings with tabs or newlines (I)}{tab}
    \begin{itemize}[label=--]
        \item In programming, \textbf{whitespace} refers to any nonprinting character, such as spaces, tabs, and end-of-line symbols.
        \item You can use whitespace to organize your output so it's easier for users to read.
    \end{itemize}
    \vspace{15pt}
    To add a tab  to your text, use the character combination \verb|\t|:
    \vspace{10pt}
    \begin{Verbatim}
>>> print("Python")
Python
>>> print("\tPython")
    Python
    \end{Verbatim}
\end{frame}


\begin{frame}[fragile]{Adding whitespace to strings with tabs or newlines (II)}{newline}
    To add a newline to your text, use the character combination \verb|\n|:
    \vspace{15pt}
    \begin{Verbatim}
>>> print("Languages:\nPython\nC\nJavaScript")
Languages:
Python
C
JavaScript
    \end{Verbatim}
\end{frame}


\begin{frame}[fragile]{Adding whitespace to strings with tabs or newlines (III)}{Combining tabs and newlines}
    \begin{itemize}[label=--]
        \item You can also combine tabs and newlines in a single string.
        \item The string \verb|"\n\t"| tells Python to move to a new line, and start the next line with a tab.
    \end{itemize}
    \vspace{15pt}
    \begin{Verbatim}
>>> print("Languages:\n\tPython\n\tC\n\tJavaScript")
Languages:
    Python
    C
    JavaScript
    \end{Verbatim}
\end{frame}


\begin{frame}{Stripping whitespace (I)}
    \begin{itemize}[label=--]
        \item Extra whitespace can be confusing in your programs.
        \item To programmers, \texttt{'python'} and \texttt{'python '} look pretty much the same.
        \item But to a program, they are two different strings.
        \item Python detects the extra space in \texttt{'python '} and considers it significant unless you tell it otherwise.
        \item It's important to think about whitespace, because often you'll want to compare two strings to determine whether they are the same.
        \item For example, when checking people's usernames when they log in to a website.
    \end{itemize}
\end{frame}


\begin{frame}[fragile]{Stripping whitespace (II)}
    To ensure that no whitespace exists at the right end of a string, use the \texttt{rstrip()} method.
    \vspace{10pt}
    \begin{Verbatim}
>>> favorite_language = 'python '
>>> favorite_language
'python '
>>> favorite_language.rstrip()
'python'
>>> favorite_language
'python '
    \end{Verbatim}
\end{frame}


\begin{frame}[fragile]{Stripping whitespace (III)}
    To remove the whitespace from the string permanently, you have to associate the stripped value with the variable name:
    \vspace{10pt}
    \begin{Verbatim}
>>> favorite_language = 'python '
>>> favorite_language = favorite_language.rstrip()
>>> favorite_language
'python'
    \end{Verbatim}
\end{frame}


\begin{frame}[fragile]{Stripping whitespace (IV)}
    You can also strip whitespace from the left side of a string using the \texttt{lstrip()} method, or from both sides at once using \texttt{strip()}:
    \vspace{10pt}
    \begin{Verbatim}
>>> favorite_language = ' python '
>>> favorite_language.rstrip()
' python'
>>> favorite_language.lstrip()
'python '
>>> favorite_language.strip()
'python'
    \end{Verbatim}
\end{frame}


\begin{frame}[fragile]{Avoiding syntax errors with strings (I)}
    A \textbf{syntax error} occurs when Python doesn't recognize a section of your program as valid Python code. \\
    \vspace{15pt}
    For example, if you use an apostrophe within singles quotes, you'll produce an error. \\
    \vspace{10pt}
    This happens because Python interprets everything between the first single quote and the apostrophe as a string. \\
    \vspace{10pt}
    It then tries to interpret the rest of the text as Python code, which causes errors.
\end{frame}


\begin{frame}[fragile]{Avoiding syntax errors with strings (II)}
    Here's how to use single and double quotes correctly.
    Save this program as \textit{apostrophe.py} and then run it:
    \vspace{10pt}
    \begin{Verbatim}
message = "One of Python's strengths is its diverse community."
print(message)
    \end{Verbatim}
    \vspace{15pt}
    The apostrophe appears inside a set of double quotes, so the Python interpreter has no trouble reading the string correctly:
    \vspace{10pt}
    \begin{Verbatim}
One of Python's strengths is its diverse community.
    \end{Verbatim}
\end{frame}


\begin{frame}[fragile]{Avoiding syntax errors with strings (III)}
    However, if you use single, quotes, Python can't identify where the string should end:
    \vspace{10pt}
    \begin{Verbatim}
message = 'One of Python's strengths is its diverse community.'
print(message)
    \end{Verbatim}
    \vspace{15pt}
    You'll see the following output:
    \vspace{10pt}
    \begin{Verbatim}[fontsize=\footnotesize]
  File "/Users/cubby/code/work/cs-class/apostrophe.py", line 1
    message = 'One of Python's strengths is its diverse community.'
                             ^
SyntaxError: invalid syntax
    \end{Verbatim}
\end{frame}


\begin{frame}{Try It Yourself (I)}
    \textbf{personal\_message.py} \\
    Use a variable to represent a person’s name, and print a message to that person.
    Your message should be simple, such as, ``Hello Eric, would you like to learn some Python today?'' \\
    \vspace{10pt}
    \textbf{name\_cases.py} \\
    Use a variable to represent a person’s name, and then print that person’s name in lowercase, uppercase, and title case. \\
    \vspace{10pt}
    \textbf{famous\_quote\_1.py} \\
    Find a quote from a famous person you admire.
    Print the quote and the name of its author.
    Your output should look something like the following, including the quotation marks: \\
    \textit{Albert Einstein once said, ``A person who never made a mistake never tried anything new.''} \\
\end{frame}


\begin{frame}[fragile]{Try It Yourself (II)}
    \textbf{famous\_quote\_2.py} \\
    Repeat \textit{famous\_quote\_1.py}, but this time, represent the famous person’s name using a variable called \texttt{famous\_person}.
    Then compose your message and represent it with a new variable called \texttt{message}.
    Print your message. \\
    \vspace{10pt}
    \textbf{stripping\_names.py} \\
    Use a variable to represent a person’s name, and include some whitespace characters at the beginning and end of the name.
    Make sure you use each character combination, \verb|\t| and \verb|\n|, at least once. \\
    \vspace{5pt}
    Print the name once, so the whitespace around the name is displayed. Then print the name using each of the three stripping functions, \texttt{lstrip()}, \texttt{rstrip()}, and \texttt{strip()}.
\end{frame}


\end{document}
