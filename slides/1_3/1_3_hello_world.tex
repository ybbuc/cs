\documentclass[aspectratio=169]{beamer}
% Themes ---------------------------------------------------------------------
\usetheme{default}
\usefonttheme{professionalfonts}
% Packages -------------------------------------------------------------------
\usepackage{graphicx}
\usepackage{enumitem}
\usepackage{stmaryrd}
\usepackage[most]{tcolorbox}
\usepackage{booktabs}
\usepackage{import}
\usepackage{adjustbox}
\usepackage{MnSymbol}
\usepackage{fancyvrb}
% Tables ---------------------------------------------------------------------
\renewcommand{\arraystretch}{1.3}
% enumitem -------------------------------------------------------------------
\setlist{itemsep=10pt}
% beamer ---------------------------------------------------------------------
\setbeamerfont{caption}{size=\small}
% Title & author -------------------------------------------------------------
\title{1.3 Running a Hello World Program}
\subtitle{Getting Started}
\author{Jakob Wells}

\begin{document}
\date{\today}


\begin{frame}
    \titlepage{}
\end{frame}


\begin{frame}[fragile]{Running \texttt{hello\_world.py} (I)}
    Before you write your first program, make a folder called \texttt{python\_work} somewhere on your system for your projects. \\
    \vspace{15pt}
    It's best to use lowercase letters and underscores for spaces in file and folder names, because Python uses these naming conventions.\\
    \vspace{15pt}
    Open Visual Studio Code, and save an empty Python file (\textbf{File} \(\blacktriangleright\) \textbf{Save As}) called \texttt{hello\_world.py} in your \texttt{python\_work} folder. \\
\end{frame}


\begin{frame}[fragile]{Running \texttt{hello\_world.py} (II)}
    The extension \texttt{.py} tells Visual Studio Code that the code in your file is written in Python, which tells it how to run the program and highlight the text in a helpful way. \\
    \vspace{15pt}
    After you've saved your file, enter the following line in the text editor:
    \vspace{5pt}
    \begin{Verbatim}
print("Hello Python world!")
    \end{Verbatim}
    \vspace{15pt}
    You can run your program by selecting \textbf{Run} \(\blacktriangleright\) \textbf{Run Without Debugging} in the menu or by pressing \texttt{CTRL-F5}. \\
\end{frame}


\begin{frame}[fragile]{Running \texttt{hello\_world.py} (III)}
    A terminal screen should appear at the bottom of the Visual Studio code window showing the following output:
    \vspace{5pt}
    \begin{Verbatim}
Hello Python world!
[Finished in 0.1s]
    \end{Verbatim}
    \vspace{15pt}
    If you don't see this output, something might have gone wrong in your program.
        \begin{itemize}[label=\(\blacktriangleright\),itemsep=5pt]
            \item Check every character on the line you entered.
            \item Did you accidentally capitalize \texttt{print}?
            \item Did you forget one of the quotation marks or parentheses?\\
        \end{itemize}
    \vspace{15pt}
    Programming languages expect very specific syntax, and if you don't provide that, you'll get errors.
\end{frame}


\begin{frame}[fragile]{Running Python programs from a terminal}{On Windows}
    You can use the terminal command \texttt{cd}, for \textit{change directory}, to navigate through your filesystem in a command window.
    The command \texttt{dir}, for \textit{directory}, shows you all the files that exist in the current directory. \\
    \vspace{10pt}
    Open a new terminal window and enter the following commands to run \textit{hello\_world.py}:
    \vspace{5pt}
    \begin{Verbatim}
C:\> cd Desktop\python_work
C:\Desktop\python_work> dir
hello_world.py
C:\Desktop\python_work> python hello_world.py
Hello Python world!
    \end{Verbatim}
    \vspace{10pt}
    Most of your programs will run fine directly from your editor.
    But as your work becomes more complex, you'll want to run some of your programs from a terminal.
\end{frame}


\begin{frame}[fragile]{Running Python programs from a terminal}{On macOS}
    You can use the terminal command \texttt{cd}, for \textit{change directory}, to navigate through your filesystem in a terminal session.
    The command \texttt{ls}, for \textit{list}, shows you all the nonhidden files that exist in the current directory. \\
    \vspace{10pt}
    Open a new terminal window and enter the following commands to run \textit{hello\_world.py}:
    \vspace{5pt}
    \begin{Verbatim}
~$ cd Desktop/python_work/
~/Desktop/python_work$ ls
hello_world.py
~/Desktop/python_work$ python hello_world.py
Hello Python world!
    \end{Verbatim}
    \vspace{10pt}
    It's that simple.
    You just use the \texttt{python3} command to run Python programs.
\end{frame}


\begin{frame}[fragile]{What really happens when you run \texttt{hello\_world.py} (I)}
    As it turns out, Python does a fair amount of work, even when it runs a simple program like the one contained in \texttt{hello\_world.py}:
    \vspace{5pt}
    \begin{Verbatim}
print("Hello Python world!") 
    \end{Verbatim}
    \vspace{15pt}
    When you run this code, you should see this output:
    \vspace{5pt}
    \begin{Verbatim}
Hello Python world!
    \end{Verbatim}
    \vspace{15pt}
    When you run the file \texttt{hello\_world.py}, the ending \texttt{.py} indicates that the file is a Python program. \\
\end{frame}


\begin{frame}[fragile]{What really happens when you run \texttt{hello\_world.py} (II)}
    Your editor then runs the file through the \textit{Python interpreter}, which reads through the program and determines what each word in the program means. \\
    \vspace{15pt}
    For example, when the interpreter sees the word \texttt{print} followed by parentheses, it prints to the screen whatever is inside the parentheses. \\
\end{frame}


\begin{frame}{Syntax highlighting}
 
    As you write your programs, your editor highlights different parts of your program in different ways. \\
    \vspace{15pt}
    For example, it recognizes that \texttt{print()} is the name of a function and displays that word in one color. \\
    \vspace{15pt}
    It recognizes that \texttt{"Hello Python world!"} is not Python code and displays that phrase in a different color. \\
    \vspace{15pt}
    This feature is called \textit{syntax highlighting} and is quite useful as you start to write your own programs.
\end{frame}


\end{document}
