\documentclass[aspectratio=169]{beamer}
% themes ---------------------------------------------------------------------
\usetheme{default}
\usefonttheme{professionalfonts}
% packages -------------------------------------------------------------------
\usepackage{graphicx}
\usepackage{enumitem}
\usepackage{stmaryrd}
\usepackage[most]{tcolorbox}
\usepackage{booktabs}
\usepackage{import}
\usepackage{adjustbox}
\usepackage{fancyvrb}
% tables ---------------------------------------------------------------------
\renewcommand{\arraystretch}{1.3}
% enumitem -------------------------------------------------------------------
\setlist{itemsep=10pt}
% beamer ---------------------------------------------------------------------
\setbeamerfont{caption}{size=\small}
% title & author -------------------------------------------------------------
\title{Chapter 3 Introducing Lists}
\author{Jakob Wells}

\begin{document}
\date{\today}


\begin{frame}
    \titlepage{}
\end{frame}


\begin{frame}[fragile]{Lists (I)}
    \begin{itemize}[label=--,itemsep=15pt]
        \item A \textbf{list} is a collection of items in a particular order.
        \item You can make a list that includes the letters of the alphabet, the digits from 0 to 9, or the names of all the people in your family.
        \item Because a list usually contains more than one element, it's a good idea to make the name of your list plural, such as \texttt{letters}, \texttt{digits}, or \texttt{names}.
    \end{itemize}
\end{frame}


\begin{frame}[fragile]{Lists (II)}
    In Python, square brackets \texttt{[ ]} indicate a list, and individual elements in the list are separated by commas.
    \vspace{15pt}
    \begin{verbatim}
animals = ['cat', 'dog', 'lizard', 'bird']
print(animals)
    \end{verbatim}
\end{frame}


\begin{frame}[fragile]{Accessing elements in a list (I)}
    \begin{itemize}[label=--,itemsep=15pt]
        \item Lists are ordered collections, so you can access any element in a list by telling Python the position, or index, of the item desired.
        \item To access an element in a list, write the name of the list followed by the index of the item enclosed in square brackets.
    \end{itemize}
    \vspace{15pt}
    \begin{verbatim}
animals = ['cat', 'dog', 'lizard', 'bird']
print(animals[0])
    \end{verbatim}
\end{frame}


\begin{frame}[fragile]{Accessing elements in a list (II)}
    \begin{itemize}[label=--,itemsep=15pt]
        \item You can also use the string methods from Chapter 2 on any element in this list.
        \item For example, you can format the element \texttt{'cat'} more neatly by using the \texttt{title()} method:
    \end{itemize}
    \vspace{15pt}
    \begin{verbatim}
animals = ['cat', 'dog', 'lizard', 'bird']
print(animals[0].title())
    \end{verbatim}
\end{frame}


\begin{frame}[fragile]{Index positions start at 0, not 1}
    \begin{itemize}[label=--,itemsep=15pt]
        \item Python considers the first item in a list to be at position 0, not position 1.
        \item The second item in a list has an index of 1.
        \item Using this counting system, you can get any element you want from a list by subtracting one from its position in the list. 
    \end{itemize}
    \vspace{15pt}
    \begin{verbatim}
animals = ['cat', 'dog', 'lizard', 'bird']
print(animals[1])
print(animals[3])
    \end{verbatim}
\end{frame}


\begin{frame}[fragile]{Accessing the last element (I)}
    \begin{itemize}[label=--,itemsep=15pt]
        \item Python has a special syntax for accessing the last element in a list.
        \item This is useful because you’ll often want to access the last items in a list without knowing exactly how long the list is.
    \end{itemize}
    \vspace{15pt}
    \begin{verbatim}
animals = ['cat', 'dog', 'lizard', 'bird']
print(animals[-1])
    \end{verbatim}
\end{frame}


\begin{frame}[fragile]{Accessing the last element (II)}
    \begin{itemize}[label=--,itemsep=15pt]
        \item The index -2 returns the second item from the end of the list, the index -3 returns the third item from the end, and so forth.
        \item By asking for the item at index \texttt{-1}, Python always returns the last item in the list.
    \end{itemize}
    \vspace{15pt}
    \begin{table}
    \centering
        \begin{tabular}{rcccc}
                                    & \texttt{'cat'} & \texttt{'dog'} & \texttt{'lizard'} & \texttt{'bird'} \\
            \textbf{index}          & 0              & 1              & 2                 & 3               \\
            \textbf{negative index} & -4             & -3             & -2                & -1              \\
        \end{tabular}
    \end{table}   
\end{frame}


\begin{frame}[fragile]{Using individual values from a list}
    \begin{itemize}[label=--,itemsep=15pt]
        \item We can use f-strings to create a message based on a value from a list.
    \end{itemize}
    \vspace{15pt}
    \begin{verbatim}
animals = ['cat', 'dog', 'lizard', 'bird']
message = f"My first pet was a {animals[0]}."

print(message)
    \end{verbatim}
\end{frame}


\begin{frame}[fragile]{classwork\_05}
    \textbf{names.py} \\
    Store the names of a few of your friends in a list called \texttt{names}.
    Print each person’s name by accessing each element in the list, one at a time. \\
    \vspace{10pt}
    \textbf{greetings.py} \\
    Start with the list you used in names.py, but instead of just printing each person’s name, print a message to them.
    The text of each message should be the same, but each message should be personalized with the person’s name. \\
    \vspace{10pt}
    \textbf{my\_list.py} \\
    Think of your favorite mode of transportation, such as a motorcycle or a car, and make a list that stores several examples.
    Use your list to print a series of statements about these items, such as ``\texttt{I would like to own a Honda motorcycle.}''
\end{frame}


\begin{frame}[fragile]{Modifying elements in a list}
    \begin{itemize}[label=--,itemsep=15pt]
        \item We can use f-strings to create a message based on a value from a list.
    \end{itemize}
    \vspace{15pt}
    \begin{verbatim}
animals = ['cat', 'dog', 'lizard', 'bird']
message = f"My first pet was a {animals[0]}."

print(message)
    \end{verbatim}
\end{frame}


\end{document}
