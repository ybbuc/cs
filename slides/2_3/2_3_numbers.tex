\documentclass[aspectratio=169]{beamer}
% themes ---------------------------------------------------------------------
\usetheme{default}
\usefonttheme{professionalfonts}
% packages -------------------------------------------------------------------
\usepackage{graphicx}
\usepackage{enumitem}
\usepackage{stmaryrd}
\usepackage[most]{tcolorbox}
\usepackage{booktabs}
\usepackage{import}
\usepackage{adjustbox}
\usepackage{fancyvrb}
% tables ---------------------------------------------------------------------
\renewcommand{\arraystretch}{1.3}
% enumitem -------------------------------------------------------------------
\setlist{itemsep=10pt}
% beamer ---------------------------------------------------------------------
\setbeamerfont{caption}{size=\small}
% title & author -------------------------------------------------------------
\title{2.3 Numbers}
\subtitle{Variables and Simple Data Types}
\author{Jakob Wells}

\begin{document}
\date{\today}


\begin{frame}
    \titlepage{}
\end{frame}


\begin{frame}[fragile]{Integers}
    You can add (\texttt{+}), subtract (\texttt{-}), multiply (\texttt{*}), and divide (\texttt{/}) integers in Python.
    \vspace{10pt}
    \begin{verbatim}
>>> 2 + 3
5
>>> 3 - 2
1
>>> 2 * 3
6
>>> 3 / 2
1.5
    \end{verbatim}
    \vspace{25pt}
    In a terminal session, Python simply returns the result of the operation.
\end{frame}


\begin{frame}[fragile]{Exponents}
    Python uses two multiplication symbols (\texttt{**}) to represent exponents:
    \vspace{10pt}
    \begin{verbatim}
>>> 3 ** 3
27
>>> 3 ** 3
27
>>> 10 ** 6
1000000
    \end{verbatim}
\end{frame}


\begin{frame}[fragile]{Order of operations}
    Python supports the order of operations.
    \vspace{10pt}
    \begin{verbatim}
>>> 2 + 3 * 4
14
>>> (2 + 3) * 4
20
>>>
    \end{verbatim}
    \vspace{25pt}
    You can also use parentheses to modify the order of operations.
\end{frame}


\begin{frame}[fragile]{Floats (I)}
    Python calls any number with a decimal point a \textbf{float}.
    \vspace{10pt}
    \begin{verbatim}
>>> 0.1 + 0.1
0.2
>>> 0.2 + 0.2
0.4
>>> 2 * 0.1
0.2
>>> 2 * 0.2
0.4
    \end{verbatim}
\end{frame}


\begin{frame}[fragile]{Floats (II)}
    But be aware that you can sometimes get an arbitrary number of decimal places in your answer:
    \vspace{10pt}
    \begin{verbatim}
>>> 0.2 + 0.1
0.30000000000000004
>>> 3 * 0.1
0.30000000000000004
    \end{verbatim}
    \vspace{25pt}
    Don't worry about this.
\end{frame}


\begin{frame}[fragile]{Integers and floats (I)}
    When you divide any two numbers, even if they are integers that result in a whole number, you’ll always get a float:
    \vspace{10pt}
    \begin{verbatim}
>>> 4 / 2
2.0
    \end{verbatim}
\end{frame}


\begin{frame}[fragile]{Integers and floats (II)}
    If you mix an integer and a float in any other operation, you’ll get a float as well:
    \vspace{10pt}
    \begin{verbatim}
>>> 1 + 2.0
3.0
>>> 2 * 3.0
6.0
>>> 3.0 ** 2
9.0
    \end{verbatim}
    \vspace{15pt}
    Python defaults to a float in any operation that uses a float, even if the output is a whole number.
\end{frame}


\begin{frame}[fragile]{Underscores in numbers}
    When you’re writing long numbers, you can group digits using underscores to make large numbers more readable:
    \vspace{10pt}
    \begin{verbatim}
>>> universe_age = 14_000_000_000
    \end{verbatim}
    \vspace{10pt}
    When you print a number that was defined using underscores, Python prints only the digits:
    \vspace{10pt}
    \begin{verbatim}
>>> print(universe_age)
14000000000
    \end{verbatim}
\end{frame}


\begin{frame}[fragile]{Multiple assignment}
    You can assign values to more than one variable using just a single line.
    For example, here’s how you can initialize the variables x, y, and z to zero:
    \vspace{10pt}
    \begin{verbatim}
>>> x, y, z = 0, 0, 0
    \end{verbatim}
    \vspace{15pt}
    Python defaults to a float in any operation that uses a float, even if the output is a whole number.
\end{frame}


\begin{frame}[fragile]{Constants}
    A \textbf{constant} is like a variable whose value stays the same throughout the life of a program.
    Python programmers use all capital letters to indicate a variable should be treated as a constant and never be changed:
    \vspace{10pt}
    \begin{verbatim}
MAX_CONNECTIONS = 5000
    \end{verbatim}
\end{frame}


\begin{frame}[fragile]{classwork\_04}
    \textbf{number\_eight.py} \\
    Write addition, subtraction, multiplication, and division operations that each result in the number 8.
    Be sure to enclose your operations in \texttt{print()} calls to see the results.
    You should create four lines that look like this: \\
    \vspace{10pt}
    \texttt{ print(5 + 3)}\\
    \vspace{10pt}
    Your output should simply be four lines with the number 8 appearing once on each line. \\
    \vspace{10pt}
    \textbf{favorite\_number.py} \\
    Use a variable to represent your favorite number.
    Then, using that variable, create a message that reveals your favorite number.
    Print that message.
\end{frame}


\end{document}
