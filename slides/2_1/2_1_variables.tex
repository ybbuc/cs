\documentclass[aspectratio=169]{beamer}
% Themes ---------------------------------------------------------------------
\usetheme{default}
\usefonttheme{professionalfonts}
% Packages -------------------------------------------------------------------
\usepackage{graphicx}
\usepackage{enumitem}
\usepackage{stmaryrd}
\usepackage[most]{tcolorbox}
\usepackage{booktabs}
\usepackage{import}
\usepackage{adjustbox}
\usepackage{MnSymbol}
\usepackage{fancyvrb}
% Tables ---------------------------------------------------------------------
\renewcommand{\arraystretch}{1.3}
% enumitem -------------------------------------------------------------------
\setlist{itemsep=10pt}
% beamer ---------------------------------------------------------------------
\setbeamerfont{caption}{size=\small}
% Title & author -------------------------------------------------------------
\title{2.1 Variables}
\subtitle{Variables and Simple Data Types}
\author{Jakob Wells}

\begin{document}
\date{\today}


\begin{frame}
    \titlepage{}
\end{frame}


\begin{frame}[fragile]{A variable in \texttt{hello\_world.py} (I)}
    Let's try using a variable in \texttt{hello\_world.py}.
    Add a new line at the beginning of the file, and modify the second line:
    \vspace{10pt}
    \begin{Verbatim}
message = "Hello Python world!"
print(message)
    \end{Verbatim}
    \vspace{10pt}
    Run this program to see what happens.
    You should see the same output you saw previously:
    \begin{Verbatim}
Hello Python world!
    \end{Verbatim}
    \vspace{10pt}
    We've added a \textit{variable} named \texttt{message}.
    Every variable is connected to a \textit{value}, which is the information associated with that variable. \\
    \vspace{10pt}
    In this case the value is the \texttt{"Hello Python world!"} text.
\end{frame}


\begin{frame}[fragile]{A variable in \texttt{hello\_world.py} (II)}
    Adding a variable makes  a little more work for the Python interpreter. \\
    \vspace{10pt}
    When it processes the first line, it associates the variable \texttt{message} with the \texttt{"Hello Python world!"} text.
    When it reaches the second line, it prints the value associated with \texttt{message} to the screen. \\
    \vspace{10pt}
    Let's expand on this program by modifying \textit{hello\_world.py} to print a second message.
    Add a blank line to \textit{hello\_world.py}, and then add two lines of code: \\
    \vspace{10pt}
    \begin{Verbatim}
message = "Hello Python world!"
print(message)

message = "Hello Python Crash Course world!"
print(message)
    \end{Verbatim}
\end{frame}


\begin{frame}[fragile]{A variable in \texttt{hello\_world.py} (III)}
    Now when we run \textit{hello\_world.py}, we see two lines of output:
    \vspace{10pt}
    \begin{Verbatim}
Hello Python world!
Hello Python Crash Course world!
    \end{Verbatim}
    \vspace{10pt}
    We can change the value of a variable in a program at any time, and Python will always keep track of its current value.
\end{frame}


\begin{frame}{Naming and using variables (I)}
    When using variables in Python, we need to follow a few rules and guidelines. \\
    \vspace{10pt}
    Breaking some of these rules will cause errors; other guidelines just help you write code that's easier to read and understand.
    \vspace{10pt}
    \begin{itemize}[label=--]
        \item Variable names can contain only letters, numbers, and underscores.
            \begin{itemize}[label=\(\blacktriangleright\),itemsep=5pt]
                \item They can start with a letter or an underscore, but not with a number.
                \item For instance, you can call a variable \texttt{message\_1} but not \texttt{1\_message}.
            \end{itemize}
        \item Spaces are not allowed in variable names, but underscores can be used to separate words in variables names.
            \begin{itemize}[label=\(\blacktriangleright\),itemsep=5pt]
                \item For Example, \texttt{greeting\_message} works, but \texttt{greeting message} will cause errors.
            \end{itemize}
    \end{itemize}
\end{frame}


\begin{frame}{Naming and using variables (II)}
    \begin{itemize}[label=--]
        \item Avoid using Python keywords and function names as variable names.
            \begin{itemize}[label=\(\blacktriangleright\),itemsep=5pt]
                \item That is, do not use words that Python has reserved for a particular programmatic purpose, such as the word \texttt{print}.
            \end{itemize}
        \item Variable names should be short, but descriptive.
            \begin{itemize}[label=\(\blacktriangleright\),itemsep=5pt]
                \item For example, \texttt{name} is better than \texttt{n}, \texttt{student\_name} is better than \texttt{s\_n}, and \texttt{name\_length} is better than \texttt{length\_of\_persons\_name}.
                \item Be careful when using the lowercase letter \texttt{l} and the uppercase letter \texttt{O} because they could be confused with the numbers \texttt{1} and \texttt{0}.
            \end{itemize}
        \item For now, all your variables should be lowercase.
            You won't get errors if you use uppercase letters, but uppercase letters in variable names have special meanings that we'll discuss later.
    \end{itemize}
\end{frame}


\begin{frame}[fragile]{Avoiding name errors when using variables (I)}
    Enter the following code, including the misspelled word \texttt{mesage}:
    \vspace{10pt}
    \begin{Verbatim}
message = "Hello Computer Science student!"
print(mesage)
    \end{Verbatim}
    \vspace{10pt}
    When an error occurs in your program, the Python interpreter does its best to help you figure out where the problem is. \\
    \vspace{10pt}
    The interpreter provides a traceback when a program cannot run successfully.
    A \textbf{traceback} is a record of where the interpreter ran into trouble when trying to execute the code.
\end{frame}


\begin{frame}[fragile]{Avoiding name errors when using variables (II)}
    Here's an example of the traceback that Python provides after you've accidentally misspelled a variable's name:
    \vspace{10pt}
    \begin{Verbatim}
Traceback (most recent call last):
  File "hello_world.py", line 2, in <module>
    print(mesage)
NameError: name 'mesage' is not defined
    \end{Verbatim}
    \vspace{10pt}
    In this case it found a \textbf{name error} and reports that the variable being printed, \texttt{mesage}, has not been defined.
    Python can't identify the variable name provided. \\
    \vspace{10pt}
    A name error usually means we either forgot to set a variable's value before using it, or we made a spelling mistake when entering the variable's name.
\end{frame}


\begin{frame}[fragile]{Avoiding name errors when using variables (III)}
    The Python interpreter doesn't spellcheck your code, but it does ensure that variable names are spelled consistently. \\
    \vspace{10pt}
    For example, watch what happens when we spell \textit{message} incorrectly in another place in the code as well:
    \begin{Verbatim}
mesage = "Hello Computer Science student!"
print(mesage)
    \end{Verbatim}
    \vspace{10pt}
    In this case, the program runs successfully!
    \vspace{10pt}
    \begin{Verbatim}
Hello Computer Science student!
    \end{Verbatim}
\end{frame}


\begin{frame}{Variables are labels}
    It's best to think of variables as labels that you can assign to values. \\
    \vspace{15pt}
    You can also say that a variable references a certain value.
\end{frame}


\begin{frame}{Try It Yourself}
    \textbf{simple\_message\_1.py} \\
    Assign a message to a variable, and then print that message. \\
    \vspace{15pt}
    \textbf{simple\_message\_2.py} \\
    Assign a message to a variable, and print that message.
    Then change the value of the variable to a new message, and print the new message.
\end{frame}


\end{document}
